\documentclass[10pt, letterpaper]{article}
\usepackage[top=80pt,bottom=80pt,left=60pt,right=60pt]{geometry}
\usepackage{fancyhdr}
\usepackage{tabularx}
\usepackage{amsmath}
\usepackage{pgfplots}
\usepackage{float}
\usepackage{rotating}
\usepackage{gensymb}
\usepackage{titling}
\usepackage{graphicx}
\newcommand{\subtitle}[1]{%
  \posttitle{%
    \par\end{center}
    \begin{center}\large#1\end{center}
    \vskip0.5em}%
}
\usepackage{pgfplotstable}
\usepackage{tikz}
\usepackage[section]{placeins}
\usepackage[utf8]{inputenc}

\pgfplotsset{compat=1.11}

\begin{document}

\title{Lab 1: Airplane Lab}
\subtitle {IB Physics II Period 6, Petach}
\date{19 September 2015}
\author{Jackson Chen}
\maketitle

\section{Design}

\subsection{Research Question}
What is the relationship between the period and the height for the circular motion of an airplane moving in a circle?

\subsection{Variables}
The independent variable is the length of the string that the airplane is attached to. The dependent variables are the
period and height of the circular motion that the airplane makes. The controlled variables include the mass of the airplane,
material of the string, the timer, the laser, and the properties of the surrounding environment.

\subsection{Apparatus}
\begin{itemize}
  \item Hook
  \item Model airplane
  \item String
  \item Stopwatch (on a phone)
  \item Meter stick
  \item Laser
  \item Ring stand with clamp
\end{itemize}

\subsection{Procedure}
\begin{enumerate}
  \item Attach the model airplane to a 100cm string and connect it to a hook attached to the ceiling. Then turn the airplane on.
  \item Clamp a laser onto a ring stand and make the laser parallel to the surface that the ring stand is on.
  \item Raise or lower the laser (without changing its angle) so its light hits the moving airplane anywhere in its circular orbit.
  \item Measure the height of the airplane with respect to the ceiling by measuring the placement of the laser with respect to the ceiling with a meter stick.
  \item Use the timer to measure the amount of seconds it takes the airplane to complete three revolutions.
  \item Repeat steps 4-5 five times for five different trials.
  \item Repeat the experiment (for one trial) for strings of lengths: 90cm, 80cm, 70cm, and 60cm.
\end{enumerate}

\subsection{Diagram}
\begin{figure}[!htb]
\centering
\includegraphics[scale=0.5]{Lab1_drawing.png}
\caption{Free Body Diagram of apparatus setup}
\end{figure}

\section{Data}

\subsection{Data Collection}
\begin{table}[H]
\centering
\begin{tabularx}{\linewidth}{>{\centering\arraybackslash}X>{\centering\arraybackslash}X>{\centering\arraybackslash}X>{\centering\arraybackslash}X>{\centering\arraybackslash}X }
\hline \textbf{String length (cm)} & \textbf{Trial} & \textbf{Height (cm)} & \textbf{Time per 3 revs(s)} & \textbf{Period (s)} \\ \hline
$100 \pm 2$ & 1 &  71.30 & 4.89 & 1.63 \\ \hline
$100 \pm 2$ & 2 &  70.11 & 4.90 & 1.63 \\ \hline
$100 \pm 2$ & 3 &  69.45 & 5.03 & 1.67 \\ \hline
$100 \pm 2$ & 4 &  67.50 & 5.03 & 1.67 \\ \hline
$100 \pm 2$ & 5 &  66.98 & 5.01 & 1.67 \\ \hline
\end{tabularx}
\caption{Data for a string length of $100 \pm 2$cm. The purpose of five trials was to determine uncertainty.}
\end{table}
\begin{table}[htp]
%\bigskip
\begin{tabularx}{\linewidth}{>{\centering\arraybackslash}X>{\centering\arraybackslash}X>{\centering\arraybackslash}X>{\centering\arraybackslash}X }
\hline \textbf{String length (cm)} & \textbf{Height (cm)} & \textbf{Time per 3 revs(s)} & \textbf{Period (s)} \\ \hline
$90 \pm 2$ & 57.95 & 4.58 & 1.53 \\ \hline
$80 \pm 2$ & 46.85 & 4.51 & 1.50 \\ \hline
$70 \pm 2$ & 37.92 & 3.95 & 1.32 \\ \hline
$60 \pm 2$ & 42.24 & 3.85 & 1.28 \\ \hline
\end{tabularx}
\caption{Data for four different string lengths}
\end{table}

\subsection{Data Processing}
\subsubsection{Deriving a relationship between period and height}


\section{Conclusion}


\end{document}
