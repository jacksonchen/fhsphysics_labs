\documentclass[10pt, letterpaper]{article}
\usepackage[top=80pt,bottom=80pt,left=60pt,right=60pt]{geometry}
\usepackage{fancyhdr}
\usepackage{tabularx}
\usepackage{amsmath}
\usepackage{pgfplots}
\usepackage{float}
\usepackage{rotating}
\usepackage{gensymb}
\usepackage{titling}
\usepackage{graphicx}
\newcommand{\subtitle}[1]{%
  \posttitle{%
    \par\end{center}
    \begin{center}\large#1\end{center}
    \vskip0.5em}%
}
\usepackage{pgfplotstable}
\usepackage{tikz}
\usepackage[section]{placeins}
\usepackage[utf8]{inputenc}

\pgfplotsset{compat=1.11}

\begin{document}

\title{Lab 2: Real Pulley}
\subtitle {IB Physics II Period 6, Petach}
\date{12 September 2015}
\author{Jackson Chen}
\maketitle

\section{Goal}

To understand the concepts of rotational mechanics and dynamics.

\section{Data}

\subsection{Part 1}

\begin{table}[htp]
\centering
\begin{tabularx}{\linewidth}{>{\centering\arraybackslash}X>{\centering\arraybackslash}X }
\hline \textbf{Mass of Pulley (g)} & \textbf{Radius of Pulley (cm)} \\ \hline
200 & 7.00 \\ \hline
\end{tabularx}
\caption{Data on the Pulley used in all of the trials of Part 1}

\bigskip

\begin{tabularx}{\linewidth}{>{\centering\arraybackslash}X>{\centering\arraybackslash}X>{\centering\arraybackslash}X>{\centering\arraybackslash}X }
\hline \textbf{Trial} & \textbf{Time (s)} & \textbf{Distance (cm)} & \textbf{Linear Accel. ($\frac{m}{s^2}$)} \\ \hline
1 & 1.41 & 72.20 & 0.726 \\ \hline
2 & 1.58 & 72.12 & 0.578 \\ \hline
3 & 1.40 & 72.25 & 0.737 \\ \hline
4 & 1.46 & 72.31 & 0.678 \\ \hline
5 & 1.53 & 72.00 & 0.615 \\ \hline
\end{tabularx}
\caption{Five trials for a 5.71 g falling mass in Part 1}

\bigskip

\begin{tabularx}{\linewidth}{>{\centering\arraybackslash}X>{\centering\arraybackslash}X>{\centering\arraybackslash}X>{\centering\arraybackslash}X }
\hline \textbf{Trial} & \textbf{Time (s)} & \textbf{Distance (cm)} & \textbf{Linear Accel. ($\frac{m}{s^2}$)} \\ \hline
1 & 1.01 & 71.20 & 1.40 \\ \hline
2 & 1.06 & 71.20 & 1.27 \\ \hline
\end{tabularx}
\caption{Two trials for a 11.02 g falling mass in Part 1}

\bigskip

\begin{tabularx}{\linewidth}{>{\centering\arraybackslash}X>{\centering\arraybackslash}X>{\centering\arraybackslash}X>{\centering\arraybackslash}X }
\hline \textbf{Trial} & \textbf{Time (s)} & \textbf{Distance (cm)} & \textbf{Linear Accel. ($\frac{m}{s^2}$)} \\ \hline
1 & 0.85 & 71.25 & 1.97 \\ \hline
2 & 0.80 & 71.20 & 2.26 \\ \hline
\end{tabularx}
\caption{Two trials for a 16.50 g falling mass in Part 1}
\end{table}

\clearpage
\subsection{Part 2}

\begin{table}[htp]
\centering
\begin{tabularx}{\linewidth}{>{\centering\arraybackslash}X }
\hline \textbf{Radius of Pulley (cm)} \\ \hline
2.40 \\ \hline
\end{tabularx}
\caption{Data on the Pulley used in all of the trials of Part 2}

\bigskip

\begin{tabularx}{\linewidth}{>{\centering\arraybackslash}X>{\centering\arraybackslash}X>{\centering\arraybackslash}X>{\centering\arraybackslash}X>{\centering\arraybackslash}X }
\hline \textbf{Trial} & \textbf{Time (s)} & \textbf{Distance (cm)} & \textbf{\alpha ($\frac{rad}{s^2}$)} & \textbf{a ($\frac{m}{s^2}$)} \\ \hline
1 & 1.00 & 80.39 & 53.23 & 1.61 \\ \hline
2 & 1.00 & 78.93 & 49.74 & 1.58 \\ \hline
3 & 1.00 & 79.90 & 50.62 & 1.60 \\ \hline
4 & 1.00 & 80.21 & 53.23 & 1.60 \\ \hline
5 & 0.95 & 80.15 & 51.49 & 1.78 \\ \hline
\end{tabularx}
\caption{Five trials for a falling mass in Part 2}
\end{table}

\section{Analysis}
\subsection{Part 1}

\begin{figure}[!htb]
\centering
\includegraphics[scale=0.5]{Lab2_drawing.png}
\caption{Free Body Diagram of apparatus setup in Part 1}
\end{figure}

\clearpage

 Conducting the sum of forces and torques based on Figure 1: \\
    \[ \Sigma F = F_G - F_T = ma \]
    \[F_T = F_G - ma = m(g-a) \] \\
    % \[\Sigma \tau = I \alpha = F_{T}R = mR(g-a) \]
    % \[mR(g-a) = \frac{1}{2}MR^{2} \alpha \]
    % \[\alpha = \frac{2mR(g-a)}{MR^2} = \frac{2m(g-a)}{MR} \] \\

\begin{enumerate}
  \item Given that $\alpha = \frac{a}{r}$
      We will use the data from Table 1 and Table 2 where $m = 5.71$g to calculate $\alpha $. To find $a$, we will average the five $a$ values in Table 2: \\[3]
      \[ a = \frac{0.726 + 0.578 + 0.737 + 0.678 + 0.615}{5} = 0.667 \frac{m}{s^2} \]
      \[\alpha = \frac{0.667}{0.07} = \boxed{9.53 \frac{rad}{s^2}} \]
      % \[\alpha = \frac{2*0.0571(9.8 - 0.667)}{0.2*0.07} = \boxed{ 74.5 \frac{rad}{s^2} } \] \\

      The $\alpha $ value for $m = 11.02$g is calculated in the same way: \\[3]
      \[ a = \frac{1.40 + 1.27}{2} = 1.34 \frac{m}{s^2} \]
      \[\alpha = \frac{1.34}{0.07} = \boxed{19.1 \frac{rad}{s^2}} \]
      % \[\alpha = \frac{2*0.1102(9.8-1.34)}{0.2*0.07} = \boxed{ 133 \frac{rad}{s^2} } \] \\

      The $\alpha $ value for $m = 16.50$g is calculated in the same way: \\[3]
      \[ a = \frac{1.97 + 2.26}{2} = 2.12 \frac{m}{s^2} \]
      \[\alpha = \frac{2.12}{0.07} = \boxed{30.3 \frac{rad}{s^2}} \]
      % \[\alpha = \frac{2*0.1650(9.8-2.12)}{0.2*0.07} = \boxed{ 181 \frac{rad}{s^2} } \]
  \item From above, it was found that $F_T = m(g-a)$. We can plug the values from the different masses and accelerations into this equation to find $F_T$: \\
    For $m = 5.71$g:
      \[ F_T = 0.00571(9.8 - 0.667) = \boxed{0.0521 N} \]
    For $m = 11.02$g:
      \[ F_T = 0.01102(9.8 - 1.34) = \boxed{0.0932 N} \]
    For $m = 16.50$g:
      \[ F_T = 0.0165(9.8 - 2.12) =  \boxed{0.127 N} \]
  \item Using Newton's second law for torques, we can find $I_{expt}$:
    \[ I_{expt} = \frac{F_{T}R}{\alpha } \] \\
    For $m = 5.71$g:
      \[ I_{expt} = \frac{0.0521*0.07}{9.53} = 3.8 * 10^{-4} \] \\
    For $m = 11.02$g:
      \[ I_{expt} = \frac{0.0932*0.07}{19.1} = 3.4 * 10^{-4}\] \\
    For $m = 16.50$g:
      \[ I_{expt} = \frac{0.127*0.07}{30.3} = 2.9 * 10^{-4} \] \\
    Averaging the $I_{expt}$ values from the three masses, the $I_{expt}$ of the pulley is \boxed{3.4 * 10^{-4} kg*m^2}
  \item From the data in Table 1:
      \[I_{calc} = 0.5*0.200*(0.07)^2 = \boxed{4.9 * 10^{-4} kg*m^2} \]
  \item Frictional torque was not taken into consideration.
  \item The frictional torque is $F_{fr}R$, and it can be calculated from the following equation:
    \[ \tau _{fr} = I_{calc}\alpha - F_{T}R \] \\
    For $m = 5.71$g:
      \[ \tau _{fr} = 4.9*10^{-4}*9.53 - 0.0521*0.07 = \boxed{0.0010 N*m} \]
    For $m = 11.02$g:
      \[ \tau _{fr} = 4.9*10^{-4}*19.1 - 0.0932*0.07 = \boxed{0.0028 N*m} \]
    For $m = 16.50$g:
      \[ \tau _{fr} = 4.9*10^{-4}*30.3 - 0.127*0.07 = \boxed{0.0060 N*m} \]
  \item Sources of error in calculating frictional torque may have included imprecise measurements. For example, when measuring the time
        it took for the mass to fall a certain distance, the exact start and stop times on the stopwatch may not have corresponded with the
        actual times. In addition, when measuring the distance that the mass fell, the measurement device we were using (a meter stick) may
        not have been perfectly perpendicular to the surface we were measuring on.
\end{enumerate}

\subsection{Part 2}
We can find $\alpha $ from the equation $\alpha = \frac{a}{r}$. Therefore we need to first average the $a$ values:
\[ a = \frac{1.61 + 1.58 + 1.60 + 1.60 + 1.78}{5} = 1.634 \]
\[ \alpha _{calc} = \frac{a}{r} = \frac{1.634}{.0240} = 68.1 \frac{rad}{s^2} \] \\

The average measured $\alpha $ value from the RoMo sensor was $51.67 \frac{rad}{s^2}$ \\

The percent error is:
\[ \frac{|68.1 - 51.67|}{51.67}*100 = \boxed{31.8 \%} \]

\end{document}
